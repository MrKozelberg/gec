% document type
\documentclass[12pt]{article}

% packages
\usepackage[total={170mm,230mm}]{geometry}
\usepackage[utf8]{inputenc}
\usepackage[T1]{fontenc}
\usepackage[russian]{babel}
\usepackage{graphicx}
\usepackage{amssymb}
\usepackage{amsfonts}
\usepackage{amsmath}
\usepackage{amsthm}
\usepackage{physics}
\usepackage{nicefrac}
\usepackage{cancel}
\usepackage{hyperref}
\usepackage{cmap}

% line number
\usepackage{lineno}
\linenumbers

% definitions
\DeclareMathOperator\arctanh{arctanh}
\DeclareMathOperator\arccosh{arccosh}
\DeclareMathOperator\const{const}
\newtheorem{definition}{Опредление}[section]
\newtheorem{theorem}{Теорема}[section]
\newtheorem{axiom}{Аксиома}[section]
\newtheorem{hypothesis}{Гипотеза}[section]

\title{Нестационарная модель ГЭЦ}
\author{А.В. Козлов \and Ф.А. Кутерин \and Н.Н. Слюняев}

\begin{document}
	\maketitle

	\section{Постановка задачи}
	Требуется решить следующую систему уравнений указанными начальными и граничными условиями:
	\begin{equation}
		\begin{cases}\label{eq:1}
			&\dfrac1{4\pi}\pdv{t} \laplacian \phi + \div(\sigma\grad\phi) = \div(\vb{j^s})\\
			&\oint_{\Gamma_1} \qty(\dfrac1{4\pi}\pdv{t} \grad\phi + \sigma\grad\phi) \dd{\vb{S}} = \oint_{\Gamma_1} \vb{j^s}\dd{\vb{S}}\\
			&\eval{\phi}_{\Gamma_1} = 0, \quad \eval{\phi}_{\Gamma_2} = V\qty(t), \quad \eval{\phi}_{t=0} = \phi_0.
		\end{cases}
	\end{equation}
	$V\qty(t)$ является неизвестной функцией. Требуется найти $\phi\qty(z,t)$.

	\begin{figure}[htbp]
		\centering
		\includegraphics[width=\linewidth]{diagram-20211026.png}
		\caption{Рисунок, поясняющий обозначения, используемые в тексте.}
		\label{fig:figure1}
	\end{figure}

	\section{Некоторые замечания}
	\par Заметим, что плотность тока $\vb{j}$ складывается из плотности тока проводимости, тока смещения и тока источников
	\begin{equation}
		\vb{j} = \vb{j^s} - \sigma\grad\phi - \dfrac1{4\pi}\pdv{t} \grad \phi.
	\end{equation}
	Ток $I$ по определению записывается так:
	\begin{equation}
		I = \oint \vb{j} \dd{\vb{S}}.
	\end{equation}
	Мы работаем с некоторой широтно\---долготной сеткой, ячейки этой сетки будем нумеровать индексом $k$. $k$\---ая ячейка представляет собой цилиндр с основанием площадью $S_k$ и высотой $h_{atm}$, в такой ячейке задана проводимость $\sigma_k\qty(z,t)$\footnote{
		Вообще говоря проводимость полагается изначально зависящей от широты и долготы $$\sigma = \sigma(z,t,\Omega),$$ где через $\Omega$ обозначена совокупность широты и долготы, но, переходя к широтно\---долготной сетке, для получения параметризации проводимости в данной ячеке мы подставляем средние значения угловых географических координат $$\Omega = \Omega_k,$$ где через $\Omega_k$ обозначена совокупность средних широты и долготы для $k$\---ой ячейки. 
	} и токи источников $\vb{j^s}_k\qty(z,t)$. Аналогично глобальной плотности тока можно ввести плотность тока в $k$\---ой ячейке
	\begin{equation}
		\vb{j}_k = \vb{j^s}_k - \sigma_k\grad\phi_k - \dfrac1{4\pi}\pdv{t} \grad \phi_k,
	\end{equation}
	где через $\phi_k = \phi_k\qty(z,t)$ обозначен потенциал электрического поля в $k$\---ой ячейке. Ток в $k$\---ой ячейке тогда запишестся следующим образом:
	\begin{equation}
		I_k = \oint_{S_k} \vb{j}_k \dd{\vb{S}}. %\approx \qty(\vb{j}_k)_z \cdot S_k.
	\end{equation}

	\section{Решение задачи}
	Проинтегрируем первое уравнение системы (\ref{eq:1}) по объёму $\Omega$, ограниченному $\Gamma_1$ и $\Gamma_2$ (см. рис. \ref{fig:figure1})
	\begin{equation}
		\iiint_\Omega \div(\vb{j}) \dd{V} = 0.
	\end{equation}
	Пользуемся формулой Гаусса и, учитывая второе уравнение системы (\ref{eq:1}), получаем
	\begin{equation}
		\oint_{\Gamma_2} \vb{j} \cdot \vb{n} \dd{S} = 0.
	\end{equation}
	Отсюда сразу же следует 
	\begin{equation}\label{eq:8}
		\oint_{\Gamma\qty(r)} \vb{j} \cdot \vb{n} \dd{S} = 0,
	\end{equation}
	где $\Gamma\qty(r)$ является некоторой сферой радиуса $r$, лежащей между $\Gamma_1$ и $\Gamma_2$. Это значит, что ток не зависит от радиуса той сферы, по которой он вычисляется.
	 В плоской геометрии для широтно\---долготной сетки уравнение (\ref{eq:8}) приобретает вид
	\begin{equation}\label{eq:11}
		\sum_k I_k\qty(t) = 0 \quad \forall t,
	\end{equation}
	где $k$\---ый ток связан с потенциалом следующим образом:
	\begin{equation}\label{eq:12}
		\dfrac{I_k(t)}{S_k} = j_k(t) = j^s_k (z,t) - \dfrac{1}{4\pi} \pdv{t} \pdv{\phi_k(z,t)}{z} - \sigma_k(z,t) \pdv{\phi_k(z,t)}{z}.
	\end{equation}
	К уравнению (\ref{eq:12}) можно применить разностную схему Кранка\---Николсона для производной по времени (это обеспечит точность порядка $(\Delta t)^2$), что аппроксимирует уравнение
	\begin{equation}
		\pdv{t} \pdv{\phi_k(z,t)}{z} = 4\pi \qty(j^s_k (z,t) - \sigma_k(z,t) \pdv{\phi_k(z,t)}{z} - \dfrac{I_k(t)}{S_k})
	\end{equation}
	разностным уравнением (где черта сверху обозначает, что значение функций берётся в будущем слое времени)
	\begin{equation}\label{eq:14}
		\dfrac1{\Delta t}\qty(\pdv{\overline{\phi_k(z)}}{z} - \pdv{\phi_k(z)}{z}) = \dfrac{4\pi}{2}\qty(\overline{j^s_k (z)} - \overline{\sigma_k(z)} \pdv{\overline{\phi_k(z)}}{z} - \dfrac{\overline{I_k}}{S_k} + j^s_k (z) - \sigma_k(z) \pdv{\phi_k(z)}{z} - \dfrac{I_k}{S_k}).
	\end{equation}
	Из разностного уравнения (\ref{eq:14}) можно выразить
	\begin{equation}\label{eq:13}
			\pdv{\overline{\phi_k(z)}}{z} = \dfrac{\qty[\frac{1}{2\pi \Delta t} - \sigma_k(z)]\pdv{\phi_k(z)}{z} + j^s_k (z) + \overline{j^s_k (z)} - \frac{I_k + \overline{I_k}}{S_k}}{\frac{1}{2\pi \Delta t} + \overline{\sigma_k(z)}}.
	\end{equation}
	Интегрируем по $z$ от $0$ до $h_{atm}$ и получаем, учитывая граничное условие,
	\begin{equation}
		\overline{V} = \int_{0}^{h_{atm}} \dfrac{\frac{1}{2\pi \Delta t} - \sigma_k(z)}{\frac{1}{2\pi \Delta t} + \overline{\sigma_k(z)}} \pdv{\phi_k(z)}{z} \dd{z} + \int_{0}^{h_{atm}} \dfrac{j^s_k (z) + \overline{j^s_k (z)}}{\frac{1}{2\pi \Delta t} + \overline{\sigma_k(z)}} \dd{z} - \int_{0}^{h_{atm}} \dfrac{1}{S_k} \dfrac{I_k + \overline{I_k}}{\frac{1}{2\pi \Delta t} + \overline{\sigma_k(z)}} \dd{z}.
	\end{equation}
	Чтобы учесть условие (\ref{eq:11}), удобнее всего выразить сумму токов
	\begin{equation}\label{eq:15}
		I_k + \overline{I_k} = \dfrac{S_k}{\int_{0}^{h_{atm}} \frac{\dd{z}}{\frac{1}{2\pi \Delta t} + \overline{\sigma_k(z)}}} \qty(\int_{0}^{h_{atm}} \dfrac{\frac{1}{2\pi \Delta t} - \sigma_k(z)}{\frac{1}{2\pi \Delta t} + \overline{\sigma_k(z)}} \pdv{\phi_k(z)}{z} \dd{z} + \int_{0}^{h_{atm}} \dfrac{j^s_k (z) + \overline{j^s_k (z)}}{\frac{1}{2\pi \Delta t} + \overline{\sigma_k(z)}} \dd{z} - \overline{V}).
	\end{equation}
	Если данное выржение просуммировать по всем ячейкам, то слева будет нуль, а справа~---~некоторая громоздкая сумма. Из полученного после суммирования уравнения, можно выразить ионосферный потенциал на следующем слое по времени
	\begin{equation}\label{eq:16}
		\overline{V} = \dfrac{\sum_k \frac{S_k}{\int_{0}^{h_{atm}} \frac{\dd{z}}{\frac{1}{2\pi \Delta t} + \overline{\sigma_k(z)}}} \qty(\int_{0}^{h_{atm}} \frac{\frac{1}{2\pi \Delta t} - \sigma_k(z)}{\frac{1}{2\pi \Delta t} + \overline{\sigma_k(z)}} \pdv{\phi_k(z)}{z} \dd{z} + \int_{0}^{h_{atm}} \frac{j^s_k (z) + \overline{j^s_k (z)}}{\frac{1}{2\pi \Delta t} + \overline{\sigma_k(z)}} \dd{z})}{\sum_k \frac{S_k}{\int_{0}^{h_{atm}} \frac{\dd{z}}{\frac{1}{2\pi \Delta t} + \overline{\sigma_k(z)}}}}.
	\end{equation}
	Теперь, чтобы найти потенциал на следующем слое по времени на всех высотах $\overline{\phi_k(z)}$ нужно снова проинтегрировать (\ref{eq:13}) по высоте, только в этот раз от $0$ до $z$
	\begin{equation}
		{\overline{\phi_k(z)}} = \int_{0}^{z} \dfrac{\frac{1}{2\pi \Delta t} - \sigma_k(z)}{\frac{1}{2\pi \Delta t} + \overline{\sigma_k(z)}} \pdv{\phi_k(z)}{z} \dd{z} + \int_{0}^{z} \dfrac{j^s_k (z) + \overline{j^s_k (z)}}{\frac{1}{2\pi \Delta t} + \overline{\sigma_k(z)}} \dd{z} - \int_{0}^{z} \dfrac{1}{S_k} \dfrac{I_k + \overline{I_k}}{\frac{1}{2\pi \Delta t} + \overline{\sigma_k(z)}} \dd{z}.
	\end{equation}
	В эту формулу следует подставить выражение (\ref{eq:15}), куда подставлен ионосферный потенциал, вычисляемый по формуле (\ref{eq:16}). В итоге имеем выражение для потенциала в "{}будущем"{} слое
	\begin{equation}\label{eq:last}
	\begin{split}
		\overline{\phi_k(z)} = &\dfrac{\int_{0}^{z} \frac{\dd{z}}{\frac{1}{2\pi \Delta t} + \overline{\sigma_k(z)}}}{\int_{0}^{h_{atm}} \frac{\dd{z}}{\frac{1}{2\pi \Delta t} + \overline{\sigma_k(z)}}} \qty(\overline{V} - \int_{0}^{h_{atm}} \dfrac{\frac{1}{2\pi \Delta t} - \sigma_k(z)}{\frac{1}{2\pi \Delta t} + \overline{\sigma_k(z)}} \pdv{\phi_k(z)}{z} \dd{z} - \int_{0}^{h_{atm}} \dfrac{j^s_k (z) + \overline{j^s_k (z)}}{\frac{1}{2\pi \Delta t} + \overline{\sigma_k(z)}} \dd{z})+\\
		& + \int_{0}^{z} \dfrac{\frac{1}{2\pi \Delta t} - \sigma_k(z)}{\frac{1}{2\pi \Delta t} + \overline{\sigma_k(z)}} \pdv{\phi_k(z)}{z} \dd{z} + \int_{0}^{z} \dfrac{j^s_k (z) + \overline{j^s_k (z)}}{\frac{1}{2\pi \Delta t} + \overline{\sigma_k(z)}} \dd{z},
	\end{split}
	\end{equation}
	куда надо подставить ионосферный потенциал из формулы (\ref{eq:16}).
	\par Результурующий алгоритм решения нестационарной задачи можно описать следующим образом. Прежде всего на основании высотного распределения потенциала на текущем временном слое вычисляется ионосферный потенциал на будущем временом слое по формуле (\ref{eq:16}). Далее по формуле (\ref{eq:last}) находится высотное распределение потенциала на будущем временном слое.
	\par Заметим, что по высоте $z$ интегралы и производные следует при имплементации заменить разностными интегралами и производными (в настоящих формулах это опущено). 
\end{document}

